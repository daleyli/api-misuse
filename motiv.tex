\section{Motivating Examples}
\label{sec:motiv}

\begin{figure}[t]
	\centering
	\lstset{
		numbers=left,
		numberstyle= \tiny,
		keywordstyle= \color{blue!70},
		commentstyle= \color{red!50!green!50!blue!50},
		frame=shadowbox,
		rulesepcolor= \color{red!20!green!20!blue!20} ,
		xleftmargin=1.5em,xrightmargin=0em, aboveskip=1em,
		framexleftmargin=1.3em,
                numbersep= 3pt,
%                numbersep=-12pt,
                framexrightmargin=-3mm,
		language=Java,
                basicstyle=\scriptsize\ttfamily,
                numberstyle=\scriptsize\ttfamily,
                emphstyle=\bfseries,
                moredelim=**[is][\color{red}]{@}{@},
		escapeinside= {(*@}{@*)}
	}
\begin{lstlisting}[]
class HttpMethodDirector {
  AuthState authstate = method.(*@{\color{red}{getHostAuthState}@*)();
  if (authstate.(*@{\color{red}{isPreemptive}@*)()) {
    authstate.(*@{\color{red}{invalidate}@*)();
(*@{\color{cyan}{\quad + authstate.}@*)(*@{\color{red}{setAuthRequested(true);}@*)
  }
  Map challenges = AuthChallengeParser.parseChallenges(
  method.getResponseHeaders(WWW_AUTH_CHALLENGE));
  ...
\end{lstlisting}
        \vspace{-12pt}
        \caption{An API misuse in Apache \code{HttpClient} project}
        \vspace{-6pt}
        \label{fig:example1}
\end{figure}

Let us use real-world examples to illustrate the problem and motivate
our solution. Fig.~\ref{fig:example1} displays an API misuse in
MuBench~\cite{mudetect-msr19}, an API misuse dataset. The code shows
the usage of the API elements in \code{httpclient} of Apache.  The
intended usage of the API elements is that when
\code{Auth\-State.\-is\-Pre\-emptive\-()} returns true (i.e., the
status of the current authentication is preemptive), both
\code{Auth\-State.\-invalidate()} and
\code{Auth\-State.\-set\-Auth\-Requested(true)} must be called. The
purpose is to invalidate the current request and to set the need for
an authentication for a later operation. However, the latter call was
missing. Thus, line 5 was added to fix the~misuse.

\begin{figure}[t]
	\centering
	\lstset{
		numbers=left,
		numberstyle= \tiny,
		keywordstyle= \color{blue!70},
		commentstyle= \color{red!50!green!50!blue!50},
		frame=shadowbox,
		rulesepcolor= \color{red!20!green!20!blue!20} ,
		xleftmargin=1.5em,xrightmargin=0em, aboveskip=1em,
		framexleftmargin=1.3em,
                numbersep= 3pt,
%                numbersep=-12pt,
                framexrightmargin=-3mm,
		language=Java,
                basicstyle=\scriptsize\ttfamily,
                numberstyle=\scriptsize\ttfamily,
                emphstyle=\bfseries,
                moredelim=**[is][\color{red}]{@}{@},
		escapeinside= {(*@}{@*)}
	}
\begin{lstlisting}[]
private void authenticateHost(final HttpMethod method) {
  ...
  if (LOG.isWarnEnabled()) {
    LOG.warn(``Required credentials not available'');
    if (method.(*@{\color{red}{getHostAuthState().isPreemptive()}@*)) {
      LOG.warn(``Preemptive authentication requested but no default credentials available'');
      method.getHostAuthState().(*@{\color{red}{invalidate();}@*)
      method.getHostAuthState().(*@{\color{red}{setAuthRequested(true);}@*)
  }
}
\end{lstlisting}
        \vspace{-12pt}
        \caption{Gihub project \code{jerenkrantz}}
        \vspace{-6pt}
        \label{fig:example2}
\end{figure}

This usage scenario is one of the intended usages of the
\code{httpClient} library. As a result, the API usage repeats in other
projects. Fig.~\ref{fig:example2} shows the same usage in the Github
project \code{jerenkrantz}. Despite the differences in the details of
surrounding code, the sequence/orders of the relevant API elements
\code{getHostAuthState}, \code{isPreemptive}, \code{invalidate}, and
\code{setAuthRequested} are the same.

%\vspace{2pt}
%\noindent {\bf Observation 1} [{\em Regularity of API Usages}]. The
%designers of software libraries have the intents for developers to use
%the API elements together (including API classes, method calls, field
%accesses) in certain combinations/orders to achieve a programming
%task.

\begin{Observation}[Regularity of API usages]
The designers of software libraries have the intention for the API
elements to be used together (including API classes, method calls,
field accesses) in certain combinations/orders to achieve a
programming task.
\end{Observation}

This observation motivates a learning-based approach that could
leverage the regularity of API usages without the need of a threshold
as in the mining-based approaches. A machine learning (ML) model could
learn the API usages from existing code corpus to decide whether the
current API usage conforms or violates API usage specifications.

Compared the usages in Fig.~\ref{fig:example1} and
Fig.~\ref{fig:example2}, we can see that the surrounding code contexts
of the above relevant API elements can be different and are less
important in deciding whether the current API usage is a violation or
not. For example, the lines 7--8 in Fig.~\ref{fig:example1} for
handling response headers and lines 3,4, and 6 in
Fig.~\ref{fig:example2} to deal with logging are less crucial than the
lines with \code{getHostAuthState}, \code{isPreemptive},
\code{invalidate}, and \code{setAuthRequested}.

\begin{Observation}[API elements in API usages]
To decide an API misuse, one needs to examine the API elements
including API classes, method calls, field accesses together with the
data and control dependencies among them and other relevant program
elements.
\end{Observation}
