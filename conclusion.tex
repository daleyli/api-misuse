\section{Conclusion}
\label{sec: conclusion}

\subsection{Threats to Validity}

Our experiment was conducted only on MUBench dataset, which is for
Java code. Therefore, the results might not be generalized to other
datasets and languages. However, the methodology of data augmentation
and graph-based code representations are general.

\subsection{Conclusion}


This paper introduces a learning-based approach for API misuse
detection that avoids the need of pre-defined thresholds. To overcome
the lack of training data, we adapt two key data augmentation
techniques to apply for a graph-based representation of API usages. We
also leverage Labeled, Graph-based Convolutional Network to learn the
embeddings for API usages, which capture the key features in API
usages and be used for API misuse detection.~Our empirical evaluation
on real-world, API misuse benchmark shows that our model relatively
improves over the state-of-the-art API misuse detection from 30--35\%
and 25-30\% at the method and API~levels, respectively.
