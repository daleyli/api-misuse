\section{Related Work}
\label{sec: related}

{\tool} is related to the line of work on API misuse detection.
The most prominent direction is mining API usage patterns
and leverage them to detect API misuses.

PRMiner is a misuse detector for C~\cite{LZ05}. It encodes usages as
the set of function calls and employs frequent-itemset mining to find
patterns. Violations are strict subsets of a pattern that occur at
least ten times less frequently than the pattern. Colibri is also for
C~\cite{L07}, which improves PRMiner using Formal Concept
Analysis~\cite{GW99}. Chronicler is a misuse detector
for~C~\cite{RGJ07b}, that mines frequent call-precedence relations
from a control-flow graph, and detects missing
method calls. Jadet, a detector for Java~\cite{WZL07},
%It encodes method-call order and call receivers in usages.
%It builds a directed graph whose nodes represent method calls on a
%given object and whose edges represent control flows. From this graph,
%
derives a pair of calls for each call-order relationship. The
sets of these pairs form the input to the mining, which identifies
patterns. A violation may miss at most two properties of the violated
pattern and needs to occur at least ten times less frequently than the
pattern.
%Jadet detects missing method calls. It may detect missing loops as a
%missing call-order relation from a method call in the loop header to
%itself.

RGJ07 is a misuse detector for C~\cite{RGJ07}.  It encodes usages as
sets of properties for each variable.  Properties are comparisons to
literals, argument positions in function calls, and assignments.
%For each call, it creates a group of the property sets of the call's arguments.
%To all groups for a particular function, it applies sequence mining to learn common sequences of control-flow properties and frequent-itemset mining to identify all common sets of all other property types.
% Detection
%Subsequently,
%
It identifies violations of the common~property sequences and sets.
% Capabilities
%\RGJ is designed to detect missing conditions.
From the properties it encodes, it can detect missing \code{null} checks and missing value or state conditions.
%Since patterns contain preceding calls on arguments, it may also detect missing calls, if the respective call shares an argument with another call in the pattern.
% Performance
%The evaluation by the authors of \RGJ applied the detector to a single
%project, thereby finding violations of project-specific patterns.  The
%authors discussed several examples of actual bugs their approach
%detects, but reported no statistics on the detection performance.

Alattin is a misuse detector for Java~\cite{TX09b}, specialized in
alternative patterns for condition checks.
%For each target method \code{m}, it queries the code-search engine
%Google Code Search to find example usages.  From each example, it
%extracts a set of rules about pre- and post-condition checks on the
%receiver, the arguments, and the return value of \code{m}, e.g.,
%``\code{boolean} check on \code{return} of \code{Iterator.hasNext}
%before \code{Iterator.next}.''
It applies frequent-itemset mining on pre-/post-condition
rules to obtain patterns.
%with a minimum support of $40\%$.
For a pattern, it extracts the subset of all groups that do
not adhere to the pattern and repeats mining on that subset to obtain
infrequent patterns.
%with a minimum support of $20\%$.
%Finally, it combines all frequent and infrequent patterns for the same
%method by disjunction.
% Detection
A method has a violation if the set of rules that hold 
is not a superset of any of the alternative (combined) patterns.
%Violations are ranked by the support of the respective pattern.
% Capabilities
%Alattin, therefore, detects missing \code{null}-checks and missing value or state conditions that are ensured by checks and that do not involve literals.
%It may also detect missing method-calls that occur in checks.
%
AX09 is a misuse detector for C~\cite{AX09}, specialized in detecting
wrong error handling, realized through returning (and checking for)
error codes.
%It distinguishes normal paths, i.e., execution paths
%from the beginning of the \code{main} function to its end, and error
%paths, i.e., paths from the beginning of the \code{main} function to
%an \code{exit} or \code{return} statement in an error-handling block.
It uses push-down model checking to generate such paths as sequences
of method calls and applies frequent-subsequence mining to find
patterns.
%
CAR-Miner is a misuse detector for C++ and Java~\cite{TX09},
specialized in detecting wrong error handling.
%For each analyzed method \code{m} in a given code corpus, it queries
%the code-search engine \aName{Google Code Search} to find example
%usages.  From the examples, it builds an \emph{Exception Flow Graph}
%(EFG), i.e., a control-flow graph with additional edges for
%exceptional flow.
From Exception Flow Graph of a method, it
generates \emph{normal} call sequences that lead to the currently
analyzed call and \emph{exception} call sequences that lead from the
call along exceptional edges.  Subsequently, it mines association
rules between normal sequences and exception sequences.
% and ranks these rules by their support.
% Detection
To detect violations,
%it extracts the normal call sequence and
%the exception call sequence for the target method call. It then
it uses the learned association rules to determine the expected
exception handling and reports a violation if the actual sequence in
the target method does not include it.

GROUMiner is a misuse detector for Java~\cite{NNP+09}.  It creates a
graph-based object-usage representation for each target method.
%rules between
%normal sequences and exception sequences.
It then performs frequent-subgraph mining on sets of such graphs to
detect patterns.  When at least 90\% of all occurrences of a
sub-pattern can be extended to a larger pattern, but some cannot,
those \emph{rare} inextensible occurrences are considered as
violations. DMMC is a misuse detector for Java~\cite{MBM10},
specialized in missing method calls.  The detection is based on type
usages, i.e., sets of methods called on a given receiver type in a
given method.
%DMMC detects misuses with exactly one missing method-call.
Tikanga is a misuse detector for Java~\cite{WZ11} that builds on
Jadet.  It extends the simple call-order properties to general
Computation Tree Logic formulae on object usages. DroidAssist is a
detector for Dalvik Bytecode (Android Java)~\cite{NPVN16}.  It
generates method-call sequences from source code and learns a Hidden
Markov Model from them, to compute the likelihood of a particular call
sequence. If the likelihood is too small, the sequence is considered a
violation. MUDetect~\cite{msr19} is a misuse detector for Java that is
extended from GROUMiner with a new graph representation.

In brief, there are two key issues with existing mining-based API
misuse detectors. First, they cannot differentiate infrequent from
invalid usage because they require to set a pre-defined threshold for
frequent API usage patterns. Second, they often learn a pattern and
then report instances of alternative usages as violations.
