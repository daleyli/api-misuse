\subsection{Key Ideas}
\label{sec:key-ideas}

From the above observations, we draw the following key ideas for our
approach.

\begin{key-idea}[Deep Learning for API usages]
We leverage Machine Learning to implicitly learn co-occurring API
elements in API usages. We exploit the basis of the regularity of API
usages: the API elements regularly appearing together in API usages
have higher impact in establishing the correct API usages than the
less regular ones. Importantly, the learning-based direction helps
avoid the use of thresholds in either establishing the API usage
patterns or distinguishing between the misuses and the uncommon
usages, which is the key limitation of the mining-based API misuse
detection approaches. We ultilize the complete, compilable code
using the libraries from a large code corpus, in which all the API
elements in use are known.
\end{key-idea}


\begin{key-idea}[Data Augmentation for API Usages and Misuses]
A key challenge in using a learning-based approach for API misuse
detection is the lack of training data. Because the number of the
samples of API misuses is much smaller in comparison with the the
number of correct API usages, one would face the issue of highly
unbalanced data. To enable our deep-learning-based solution, we have
utilized a few technique for data augmentation to increase the numbers
of API misuses for model training
(Section~\ref{sec:data-augmentation}).
\end{key-idea}

\begin{key-idea}[Graph-based Neural Network Model for API Usage Representation Learning]
We design a graph-based, representation learning model to {\em learn
  the contextualized embeddings for API Usages} that integrates {\em
  data and control dependencies}, and control structures among API and
program elements. We train a Graph-based Convolution
Network~\cite{gcn} to learn the embeddings for API usages from a
graph-based API usage representation, called API Usage Graph
(AUG)~\cite{msr19}. The embeddings will capture the key features
in API usages and be used for API misuse detection.
\end{key-idea}
